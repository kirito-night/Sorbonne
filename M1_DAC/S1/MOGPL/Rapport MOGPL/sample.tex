\documentclass[twoside,11pt]{article}

\usepackage{UFrecherche}
\newcommand{\dataset}{{\cal D}}
\newcommand{\fracpartial}[2]{\frac{\partial #1}{\partial  #2}}

% Heading arguments are {Semester}{year}{Objective}{Date}{}{authors}
\ufheading{Semester 1}{2022-2023}{Bib.Review}{10.Jan.23}{UF Recherche - 22/23}{M.Rojas}

% Short headings should be running head and authors last names

\ShortHeadings{Short Title}{Nom-1, Nom-2 and Nom-3}
\firstpageno{1}

\begin{document}

\title{Title of your research}

\author{\name Prenom Nom 1 \email email1@insa-toulouse.fr \\
       \addr Department of Mechanical Engineering \\
       INSA Toulouse\\
       Toulouse, France
       \AND
       \name Prenom Nom 2 \email email2@insa-toulouse.fr \\
       \addr Department of Mechanical Engineering \\
       INSA Toulouse\\
       Toulouse, France}

%Name of your tutors
\editor{E. Marenic, P. Oumaziz and A.C. Araujo}

\maketitle

\begin{abstract}%   <- trailing '%' for backward compatibility of .sty file
This paper describes the mixtures-of-trees model, a probabilistic 
model for discrete multidimensional domains.  Mixtures-of-trees 
generalize the probabilistic trees of \citet{chow:68}
in a different and complementary direction to that of Bayesian networks.
We present efficient algorithms for learning mixtures-of-trees 
models in maximum likelihood and Bayesian frameworks. 
We also discuss additional efficiencies that can be
obtained when data are ``sparse,'' and we present data 
structures and algorithms that exploit such sparseness.
Experimental results demonstrate the performance of the 
model for both density estimation and classification. 
We also discuss the sense in which tree-based classifiers
perform an implicit form of feature selection, and demonstrate
a resulting insensitivity to irrelevant attributes.
\end{abstract}

\begin{keywords}
  Bayesian Networks, Mixture Models, Chow-Liu Trees
\end{keywords}

\section{Introduction}

Probabilistic inference has become a core technology in AI,
largely due to developments in graph-theoretic methods for the 
representation and manipulation of complex probability 
distributions~\citep{pearl:88}.  Whether in their guise as 
directed graphs (Bayesian networks) or as undirected graphs (Markov 
random fields), \emph{probabilistic graphical models} have a number 
of virtues as representations of uncertainty and as inference engines.  
Graphical models allow a separation between qualitative, structural
aspects of uncertain knowledge and the quantitative, parametric aspects 
of uncertainty...\\


% Acknowledgements should go at the end, before appendices and references

\section{Name of section 2}


 Let $u,v,w$ be discrete variables such that $v, w$ do
not co-occur with $u$ (i.e., $u\neq0\;\Rightarrow \;v=w=0$ in a given
dataset $\dataset$). Let $N_{v0},N_{w0}$ be the number of data points for
which $v=0, w=0$ respectively, and let $I_{uv},I_{uw}$ be the
respective empirical mutual information values based on the sample
$\dataset$. Then
\begin{equation}
N_{v0} \;>\; N_{w0}\;\;\Rightarrow\;\;I_{uv} \;\leq\;I_{uw}
\end{equation}

with equality only if $u$ is identically 0.
 Entropies will be denoted
by $H$. We aim to show that $\fracpartial{I_{uv}}{P_{v0}} < 0$....\\

\section{Conclusions}

Subjects in this study were able to ....

\vskip 0.2in

\bibliography{samplebib}


\end{document}